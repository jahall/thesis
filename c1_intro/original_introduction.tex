\chapter{Introduction}

In the early days of reinforcement learning \cite{SBW92} made the link that reinforcement learning can be viewed as direct adaptive control of nonlinear systems. However, the fields have continued to operate in distinct domains due to the vast majority of reinforcement learning algorithms lacking the ability to scale into large and continuous state spaces. However, the past decade has seen this change with the advent of more advanced algorithms that combine elements from the field of supervised learning for function approximation. This has led to significant overlap in the kinds of problem tackled by each field. However, currently this wealth of methods remain quite disjoint with no clear framework to determine what algorithm would be more efficacious or preferable for a given situation. This thesis seeks to remedy this!